\documentclass{amsbook}
\usepackage{listings}
\usepackage{xcolor}
\usepackage{courier}
\usepackage{latexsym}

\newtheorem{theorem}{Theorem}
\newtheorem{definition}{Definition}
\newtheorem{notation}{Notation}

\renewcommand*{\familydefault}{\rmdefault}

\definecolor{codegreen}{rgb}{0,0.6,0}
\definecolor{codegray}{rgb}{0.5,0.5,0.5}
\definecolor{codepurple}{rgb}{0.58,0,0.82}
\definecolor{backcolour}{rgb}{0.95,0.95,0.92}

\lstnewenvironment{cpp}
{
 \bigskip                    
 \lstset{
 	language=C++, % choose the language of the code.
 	frame=single,
 	tabsize=2,
 	rulesepcolor=\color{gray},
 	rulecolor=\color{black},
  backgroundcolor=\color{backcolour},   
  commentstyle=\color{codegreen},
  keywordstyle=\color{magenta},
  numberstyle=\tiny\color{codegray},
  stringstyle=\color{codepurple},
  basicstyle=\ttfamily\footnotesize,
  breakatwhitespace=false,         
  breaklines=true,                 
  captionpos=b,                    
  keepspaces=true,                 
  numbers=left,                    
  numbersep=5pt,                  
  showspaces=false,                
  showstringspaces=false,
  showtabs=false,                  
  tabsize=2
}
}
{}

\lstset{aboveskip=10pt,belowskip=10pt}

\begin{document}

\chapter{Basics}

\section{String-Literals}
\begin{cpp}
char english[] = ``A book holds a house of gold.'';
\end{cpp}

\section{Range-Based For Loop}

\begin{cpp}
for (element-type element-name : array-name) {
}
\end{cpp}

\section{Enum Class}

\begin{cpp}
  enum class Race {
    Elf,
    Dwarf,
    Halfing,
    Human,
    Hobbit
 };

  Race myrace = Race::Hobbit;

  switch(race) {
    case Race::Dwarf {
      printf(``You carry an axe.'');
      break;
    }
    case Race::Hobbit {
      printf(``Your feet are hairy.'');
      break;
    }
    default: {
      printf(``Error: unknown race!'');
    }
  }  
\end{cpp}

\section{Braced Initialization}

\begin{cpp}
int x = 0;
int x = 42;
int x = {};
int x = {42};
int x{};
int x{42};
\end{cpp}

\section{PODs}

\begin{cpp}
  struct POD {
    string name;
    int age;
    float salary;
  };

  POD pod1 {``Rick'', 50, 85e3};
  
\end{cpp}

\section{Using}

\begin{cpp}

  using real = double;

  using vertex = std::string;

  template <typename Value>
  using vector = tensor<1, Value>;

\end{cpp}

\section{Class Basics}

\begin{enumerate}
\item struct members are public by default.
\item class members are private by default.
\item class invariant: a feature of a class that is always true.

\begin{cpp}

  struct Clock {
    void addyear() {
      year++;
    }
    int year;
  };
 
\end{cpp}
  
\end{enumerate}


\chapter{Constructors}

When does C++ supply constructors?

\begin{enumerate}
\item The compiler creates a default constructor.
\item The compiler also creates a copy constructor for us by copying all data members.
\item The compiler doesn't create a default constructor if we supply any constructor.
\item The compiler creates a copy constructor if we don't supply our own, even if we've writtern other constructors.
\end{enumerate}

\section{Copy Constructor}
\begin{cpp}
  
  vector(const vector& v) : my_size{v.my_size}, data{new double[my_size]}
  {
    for (unsigned i = 0; i < my_size; i++) {
      data[i] = v.data[i];
    }
    
\end{cpp}

\chapter{Templates}

\section{Partial Specialization}

\begin{cpp}

  template <typename RealType}
  class vector< complex<RealType> >
  {
  }

  template <typename T>
  inline T abs(const std::complex<T>& x)
  {
  }

\end{cpp}

\section{Non-Type Parameters for Templates}

\begin{cpp}

  template <typename T, int Size = 3>
  
  class FSizeVector {
    
    using self = FSizeVector;

    public:
    
        using valuetype = T;

        const static int mysize = size;
        FSizeVector(int s = Size);

        self& operator=(const self& that);
        self operator+(const self& that) const;
        
  }
    
\end{cpp}


\chapter{Type Traits}

\begin{cpp}

  #include <type_traits>

  template <typename T>
  void foo(T) {
    std::cout << ``T is signed'' << std::endl;
  }

  template <class T, class = typename std::enable_if<std::is_unsigned<T>::value>::type>
  void foo(T) {
    std::cout << ``T is unsigned'' << std::endl;
  }

  struct A {};
  struct B : A {};

  template <class T, class = typename std::enable_if<std::is_base_of<A, T>::<value>::type>
  struct C : T {};
    
\end{cpp}

\end{document}




 	
